
\section{Methodology}

In order to document the hardware performance of one of the IBM QC hardware platforms, we focused on running all of the computations on the IBM 20 qubit Boeblingen quantum computing hardware platform.  \textbf{FIGURE BOEBLINGEN QUBIT LAYOUT}.    


With the cooperation of the IBM QC hardware group system administration group at Yorktown Heights, we were able to secure more than 140 hours of dedicated mode access to Boeblingen over an 8 day consecutive period in January 2021.  

Our team scheduled dedicated mode reservation time on Boeblingen each day during the time frames from 4 am to 10 am and again from 3 pm until 11 pm.  By being able to place Boeblingen in dedicated mode, we were guaranteed that no other users would have access to the machine during those time windows.  

The IBM team agreed to schedule a daily full re-calibration of the Boeblingen hardware platform each day beginning at 4 am local time.  This full re-calibration took approximately 2 hours to complete.  The IBM team also agreed to do a separate 2 qubit re-calibration of Boeblingen each day beginning at 6 pm local time.  The 2 qubit only re-calibration took approximately and hour to complete.  Both of those re-calibration proceeded during the time periods when the machine was completely quiescent from any user activity. 

After the IBM quantum hardware re-calibrations were completed we recorded the Boeblingen back-end properties.  We then began both the cycle benchmarking followed by the Transvere Ising model computations while the Boeblingen platform continued to remain in dedicated mode.  \textbf{A detailed description of the cycle benchmarking calculations can be found in the Appendix.}

The circuit diagram for the TIM (REF FIGURE WITH TIM CIRCUIT DIAGRAMS) showed that there were three pairs of two qubit CNOT gates that were the focus of this project's investigation.  The cycle benchmarking calculations were prepared so that the CNOT gates were measured in two different configuration sequences Fig.~\ref{fig:IsingTrotterCircs}. these two configurations were labelled as Circuit 1 and Circuit 2.  Both of these quantum circuits result in time evolution of a state with $\mathcal{U}=e^{-iH_{\text{OBC}}\tau}$ using Trotterization where $\tau$ is the time interval for one Trotter step.  
\begin{figure*}[!tb]
\[ {\Qcircuit @C=0.5em @R=0.5em {
     \lstick{} & \gate{R_Z(h \tau)} & \multigate{1}{R_{XX}(2 J \tau)} &\qw&\qw  & & & &\\ 
     \lstick{} & \gate{R_Z(h \tau)}  & \ghost{R_{XX}(2 J \tau)} & \multigate{1}{R_{XX}(2 J \tau)}&\qw & & & &\\
     \lstick{} &  \gate{R_Z(h \tau)} & \multigate{1}{R_{XX}(2 J \tau)}&\ghost{R_{XX}(2 J \tau)}&\qw & & & & \\
     \lstick{} &  \gate{R_Z(h \tau)} & \ghost{R_{XX}(2 J \tau)} &\qw&\qw& & & &}}
    { \Qcircuit @C=0.5em @R=0.5em {
     \lstick{} & \gate{R_Z(h \tau)} & \multigate{1}{R_{XX}(2 J \tau)} &\qw&\qw &\qw & \qw\\ 
     \lstick{} & \gate{R_Z(h \tau)}  & \ghost{R_{XX}(2 J \tau)} & \multigate{1}{R_{XX}(2 J \tau)}&\qw&\qw&\qw \\
     \lstick{} &  \gate{R_Z(h \tau)} & \qw&\ghost{R_{XX}(2 J \tau)}&\qw & \multigate{1}{R_{XX}(2 J \tau)}\\
     \lstick{} &  \gate{R_Z(h \tau)} & \qw &\qw&\qw&\ghost{R_{XX}(2 J \tau)}}
}\]
\caption{The quantum circuit for one Trotter step of the time evolution with the open boundary condition Ising model Hamiltonian. We define the quantum circuit in left (right) panel as Circuit 1 (Circuit 2).}
\label{fig:IsingTrotterCircs}
\end{figure*}

This cycle benchmarking and TIM computations were each implemented on three sets of 4 qubits located on different physical sections of the 20 qubit chip (qubits [0,1,2,3] in the upper left, qubits [6,7,11,12] in the center and qubits [16,17,18,19] on the lower right].  In the paper these are identified as Layout 1, 2 and 3 respectively.  For the cycle benchmarking portion, using the identity and Pauli Z as the set of Cliffords for the cycle benchmarking, the expectation values versus circuit sequence length were measured for circuit lengths of 2, 10 and 22 and plotted. Using the three different circuit lengths the amplitude and slope of the exponent were calculated for all 16 of the Pauli decay terms.
These calculations were repeated for the exhaustive set of two qubit pairs within each of the three different layouts.  For example, on layout 1 measurements included all of  the combination of two qubit pairs ( [0,1 and 2,3], [0,1], [1,2] and [2,3] ).  Similar measurements were taken on the CNOT pairs for Layouts 2 and 3.   Using these measurements the Pauli infidelity terms were measured and each of the error bars were calculated for every one of the 16 Pauli decay terms.  From this information the overall circuit process infidelity and error bar was computed.  Finally, using this information the quantum capacity (QCAP) was calculated versus the step size.  This QCAP bound graph showed how a dressed cycle of a circuit will have a stochastic error model whose process fidelity is estimated directly by cycle benchmarking.  From this graph the performance of the circuit implemented on the set of specific qubits on that specific hardware platform can be measured over time.  






\begin{figure*}[!tb]
\[ {\Qcircuit @C=0.5em @R=0.5em {
     \lstick{} & \gate{R_Z(h \tau)} & \multigate{1}{R_{XX}(2 J \tau)} &\qw&\qw  & & & &\\ 
     \lstick{} & \gate{R_Z(h \tau)}  & \ghost{R_{XX}(2 J \tau)} & \multigate{1}{R_{XX}(2 J \tau)}&\qw & & & &\\
     \lstick{} &  \gate{R_Z(h \tau)} & \multigate{1}{R_{XX}(2 J \tau)}&\ghost{R_{XX}(2 J \tau)}&\qw & & & & \\
     \lstick{} &  \gate{R_Z(h \tau)} & \ghost{R_{XX}(2 J \tau)} &\qw&\qw& & & &}}
    { \Qcircuit @C=0.5em @R=0.5em {
     \lstick{} & \gate{R_Z(h \tau)} & \multigate{1}{R_{XX}(2 J \tau)} &\qw&\qw &\qw & \qw\\ 
     \lstick{} & \gate{R_Z(h \tau)}  & \ghost{R_{XX}(2 J \tau)} & \multigate{1}{R_{XX}(2 J \tau)}&\qw&\qw&\qw \\
     \lstick{} &  \gate{R_Z(h \tau)} & \qw&\ghost{R_{XX}(2 J \tau)}&\qw & \multigate{1}{R_{XX}(2 J \tau)}\\
     \lstick{} &  \gate{R_Z(h \tau)} & \qw &\qw&\qw&\ghost{R_{XX}(2 J \tau)}}
}\]
\caption{The quantum circuit for one Trotter step of the time evolution with the open boundary condition Ising model Hamiltonian. We define the quantum circuit in left (right) panel as Circuit 1 (Circuit 2).}
\label{fig:IsingTrotterCircs}
\end{figure*}




\subsection{Cycle Bnchmarking}

\label{sec:cb}
We used the cycle benchmarking protocol studied in \cite{Erhard2019} and TrueQ software developed by Quantum Benchmark.



\subsection{Cycle Benchmarking Procedure?}
The measured process infidelities ($e_F$) obtained using CB protocol are presented in 


The cycle benchmarking data was obtained using the following variables. For each cycle of interest we used sequence lengths of 2, 10, and 22, respectively. The sequence length refers to the number of times the cycle of interest appears apart from state inversion. The number of random circuits used in each sequence length is 48. Each of these circuits at each sequence length were run $N_\text{shots}=128$ times. 

The process infidelity is estimated from the average value of the individual Pauli infidelities.


\subsubsection{Process Infidelity Measurements}
The process infidelity ($e_{F}$) was calculated using cycle benchmarking method for CNOT cycles seen in Fig.~\ref{fig:BoeblingenCycles}. Each row in Fig.~\ref{fig:BoeblingenCycles} corresponds to the CNOT cycles  studied for each qubit layout. There are four cycles studied for each qubit layout and we label each of these unique cycle as cycle 1, cycle 2, cycle 3 and cycle 4, respectively from left to right.
\begin{figure}[ht!]
    % \centering
    % \includegraphics[width=2.2\columnwidth]{final_plot.pdf}
    \includegraphics[scale=0.4]{BoeblingenCycles.pdf}
    \caption{The CNOT cycles used to calculate the process infidelities on IBM Q Boeblingen device.}
    \label{fig:BoeblingenCycles}
\end{figure}






\subsection{Quantum Capacity (QCap) Bound}
In Fig.s~\ref{fig:QCapCirc1}, \ref{fig:QCapCirc2}, \ref{fig:QCapCirc1and2} we present the QCap bound ($e_{IU}$) values calculated as a function of number of Trotter steps for circuits 1, circuit 2 and circuits 1 and 2 together. The data in Fig.~\ref{fig:QCapCirc1} demonstrates that qubit layout [6, 7, 12, 11] on Boeblingen quantum hardware performs the best for almost all runs and in each day of the experiment for Circuit 1. 

We also studied quantum capacity (QCap) as a function of the number of Trotter steps. QCap is able to estimate an upper bound on the total variational distance (TVD) without a full quantum simulation by characterizing the error rate of each cycle in the circuit and combining the results. This upper bound assumes the circuit is being run under randomized compiling (RC). Under RC, each dressed cycle of a circuit will have a stochastic error model whose process fidelity is estimated directly by CB. In this setting, errors accumulate in a circuit in a predictable way: the process fidelity of the circuit is the product of the process fidelity of each cycle. This parameter is estimated as this product, with error bars derived from standard propagation of uncertainty techniques. {\color{red}{KYA: Cite QB website.}}





%\begin{figure*}[ht!]
    % \centering
    % \includegraphics[width=2.2\columnwidth]{final_plot.pdf}
%    \includegraphics[scale=0.5]{QCapBoundForEachDay_Circ1.pdf}
%    \caption{QCap bounds calculated using IBM Q Boeblingen hardware as a function of number of Trotter steps for circuit 1, layouts $[0,1,2,3],[6,7,12,11],[16,17,18,19]$, runs 1, 2, and 3 between days 01/23-31/2021 are presented.}
%    \label{fig:QCapCirc1}
%\end{figure*}
%\begin{figure*}[ht!]
    % \centering
    % \includegraphics[width=2.2\columnwidth]{final_plot.pdf}
%    \includegraphics[scale=0.5]{QCapBoundForEachDay_Circ2.pdf}
 %   \caption{QCap bounds calculated using IBM Q Boeblingen hardware as a function of Trotter steps for circuit 2, layouts $[0,1,2,3],[6,7,12,11],[16,17,18,19]$, runs 1, 2, and 3 between days 01/23-31/2021 are presented.}
%    \label{fig:QCapCirc2}
%\end{figure*}

%\begin{figure*}[ht!]
    % \centering
    % \includegraphics[width=2.2\columnwidth]{final_plot.pdf}
%    \includegraphics[scale=0.5]{QCapBoundForEachDay_Circ1andCirc2.pdf}
%    \caption{QCap bounds calculated using IBM Q Boeblingen hardware as a function of number of Trotter steps for circuit 1 and 2 are compared for layouts $[0,1,2,3],[6,7,12,11],[16,17,18,19]$, runs 1, 2, and 3 between days 01/23-31/2021 are presented.}
%    \label{fig:QCapCirc1and2}
%\end{figure*}

The parameters used for QCap measurements is as follows. Due to limited access to the dedicated mode on Boeblingen device we use sequence lengths of 4 and 16. The number of random circuits in this case is 30 and each of these circuits were run $N_{\text{shots}}=128$. 

