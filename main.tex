
\documentclass[aps,pra,reprint,showpacs,nofootinbib,superscriptaddress]{revtex4-1}
%%%%%%%%%%%%%%%%%%%%%%%%%%%%%%%%%%%%%%%%%%%%%%%%%%%%%%%%%%%%%%%%%%%%%%%%%%%%%%%%%%%%%%%%%%%%%%%%%%%%%%%%%%%%%%%%%%%%%%%%%%%%%%%%%%%%%%%%%%%%%%%%%%%%%%%%%%%%%%%%%%%%%%%%%%%%%%%%%%%%%%%%%%%%%%%%%%%%%%%%%%%%%%%%%%%%%%%%%%%%%%%%%%%%%%%%%%%%%%%%%%%%%%%%%%%%
\input{Qcircuit}
% \usepackage{tikz}
% \usetikzlibrary{quantikz}
\usepackage{graphicx,comment}
\usepackage{amsfonts}
\usepackage{amsmath,amssymb}
%\usepackage[varg]{txfonts}
\usepackage{bm}
\usepackage{color}
% \usepackage{epstopdf}
% \epstopdfsetup{outdir=./Figures/}
% \usepackage{epsfig}
% \usepackage[font=small,
%   justification=justified,
%   format=plain]{caption} % 'format=plain' avoids hanging indentation
% \usepackage{subfig}
%\usepackage{float}
%\usepackage[compat=1.1.0]{tikz-feynman}
%\usepackage{a4wide,amssymb,float}
\usepackage{float}
\usepackage{verbatim}
\usepackage{hyperref}
\usepackage{multirow}
\usepackage[table,xcdraw]{xcolor}
\usepackage{tabularx}
\usepackage{wrapfig}
% \usepackage{ragged2e}
% \usepackage{subcaption}
% \usepackage[skip=2pt]{caption}
\hypersetup{
     colorlinks   = true,
     citecolor    = blue
}
\usepackage{lineno}
% \usepackage[thicklines]{cancel}
\usepackage{color, colortbl}
% \definecolor{LightCyan}{rgb}{0.88,1,1}
% \usepackage{url}
% \usepackage{dcolumn}
%
%  package for subfiles
%
\usepackage{subfiles}
% %
% \usepackage{lineno}
\newcommand{\todo}[1]{\textcolor{red}{#1}}    
% \usepackage{graphicx}
\def\ho{\tilde{\omega}}
\def\tio{\hat{\omega}}
\def\tp{\tilde{p}}
%%%%%%%%%%%%%%%%%%%%%%%%%%%%%%%%%%%%%%%%%%%%%%%%
\def\de{\partial}
% \def\a{\alpha}
\def\b{\beta}
\def\g{\gamma}
\def\G{\Gamma}
\def\d{\delta}
\def\D{\Delta}
\def\e{\eta}
\def\f{\phi}{\rm }
\def\la{\lambda}
\def\La{\Lambda}
\def\k{\kappa}
\def\m{\mu}
\def\n{\nu}
\def\r{\rho}
%\def\o{\omega}
\def\p{\pi}
\def\s{\sigma}
\def\S{\Sigma}
\def\t{\tau}
\def\ep{\epsilon}
\def\th{\theta}
\def\z{\zeta}
\def\x{\chi}
%%%%%%%%%%%%%%%%%%%%%%%%%%%%%%%%%%%%%%%%%%%%%%%%%
\def\be{\begin{equation}}
 \def\ee{\end{equation}}
 \def\bea{\begin{eqnarray}}
 \def\eea{\end{eqnarray}}
 % ------- Define Greek Lowercase --------
%  \def\a{\alpha}
 \def\b{\beta}
 \def\g{\gamma}
 \def\d{\delta}
 %\def\o{\omega}
 \def\s{\sigma}
 % ------- Define Greek Uppercase --------
\def\G{\Gamma}
\def\L{\Lambda}

\def\ho{\tilde{\omega}}
\def\tio{\hat{\omega}}
\def\tp{\tilde{p}}
\newcommand{\fr}{\frac}
\newcommand{\pr}{\prime}
\newcommand{\pp}{{\prime \prime}}
%\renewcommand{\CancelColor}{\color{red}}
%\newcommand{\stkout}[1]{\ifmmode\text{\sout{\ensuremath{#1}}}\else\sout{#1}\fi}
\def\pa{\partial}
\def\A{\mathcal{A}}
\def\B{\mathcal{B}}
\def\F{\mathcal{F}}
\def\H{\mathcal{H}}
\def\K{\mathcal{K}}
\def\N{\mathcal{N}}
\def\O{\mathcal{O}}
\def\R{\mathcal{R}}
\def\W{\mathcal{W}}
% \def\2{\frac{1}{2}}
% \def\4{\frac{1}{4}}

\def\ap{a_+}
\def\am{a_-}
%\newcommand{\bra}[1]{\left\langle #1 \right|}
%\newcommand{\ket}[1]{\left| #1 \right\rangle}
\newcommand{\expect}[1]{\langle {#1} \rangle}
\newcommand{\braket}[2]{\left\langle #1 \vert #2 \right\rangle}

\def\nn{\nonumber}


% \def\np#1{{Nucl.\ Phys.\ B \bf #1}}
% \def\pl#1{{Phys.\ Lett.\ B \bf #1}}
% \def\PLA#1{{Phys.\ Lett.\ A \bf #1}}
% \def\PR#1{{Phys.\ Rev.\ D \bf #1}}
% \def\PRL#1{{Phys.\ Rev.\ Lett.\ \bf #1}}
% \def\cm#1{{Commun.\ Math.\ Phys.\ \bf #1}}
% \def\mpl#1{{Mod.\ Phys.\ Lett.\ A \bf #1}}
% \def\cpc#1{{Comp.\ Phys.\ Comm.\ \bf #1}}
% \def\anp#1{{Annals Phys.\ \bf #1}}
% \def\rmp#1{{Rev.\ Mod.\ Phys.\ \bf #1}}
% \def\cqg#1{{Class.\ Quant.\ Grav.\ \bf #1}}
% \def\gen{\mathrm{g}}

% \catcode`\@=11

% %       This causes equations to be numbered by section

% %\@addtoreset{equation}{section}
% %\def\theequation{\arabic{equation}}
% %\def\theequation{\thesection.\arabic{equation}}

% \def\@normalsize{\@setsize\normalsize{15pt}\xiipt\@xiipt
% \abovedisplayskip 14pt plus3pt minus3pt%
% \belowdisplayskip \abovedisplayskip
% \abovedisplayshortskip  \z@ plus3pt%
% \belowdisplayshortskip  7pt plus3.5pt minus0pt}
% \def\small{\@setsize\small{13.6pt}\xipt\@xipt
% \abovedisplayskip 13pt plus3pt minus3pt%
% \belowdisplayskip \abovedisplayskip
% \abovedisplayshortskip  \z@ plus3pt%
% \belowdisplayshortskip  7pt plus3.5pt minus0pt
% \def\@listi{\parsep 4.5pt plus 2pt minus 1pt
%             \itemsep \parsep
%             \topsep 9pt plus 3pt minus 3pt}}

% \def\underline#1{\relax\ifmmode\@@underline#1\else
%         $\@@underline{\hbox{#1}}$\relax\fi}
% \@twosidetrue \relax


% \catcode`@=12

% %       set page size
% %\evensidemargin 0.0in \oddsidemargin 0.0in \topmargin -0.2in
% %\textwidth 6.4in \textheight 8.9in \headsep .50in

% %       reset section commands

% %       reset section commands

% \catcode`\@=11
% \def\lover#1{
%       \raisebox{1.3ex}{\rlap{$\leftarrow$}} \raisebox{ 0ex}{$#1$}}
% \def\section{\@startsection{section}{1}{\z@}{3.5ex plus 1ex minus
%   .2ex}{2.3ex plus .2ex}{\large\bf}}
% %\def\thesection{\arabic{section}.}
% %\def\thesubsection{\arabic{section}-\arabic{subsection}.}
% %       reset the page style
% \def\FERMIPUB{}
% \def\FERMILABPub#1{\def\FERMIPUB{#1}}
% \def\ps@headings{\def\@oddfoot{}\def\@evenfoot{}
% \def\@oddhead{\hbox{}\hfill
%         \makebox[.5\textwidth]{\raggedright\ignorespaces --\thepage{}--
%         \hfill }}
% \def\@evenhead{\@oddhead}
% \def\subsectionmark##1{\markboth{##1}{}}
% }

% \ps@headings

% \catcode`\@=12

%%%%%%%%%%%%%%%%%%%%%%%%%%%%%%%%%%%%%%%%%%%%%%%%%%%%%%%%%%%%%%%%%%%%%%%%
\linenumbers
\begin{document}
\title {Impacts of NISQ Quantum Hardware Errors on Quantum Computing Application Performance}

%\title{Studying the Ising Model Trotterization with Cycle Benchmarking and Randomized Compiling}
%
{\footnote{ \textbf{SHOULD THIS GO IN THE ACKNOWLEDGMENTS? DOES IT BELONG HERE ON PAGE 1} This manuscript has been authored by UT-Battelle, LLC, under Contract No. DE-AC0500OR22725 with the U.S. Department of Energy. The United States Government retains and the publisher, by accepting the article for publication, acknowledges that the United States Government retains a non-exclusive, paid-up, irrevocable, world-wide license to publish or reproduce the published form of this manuscript, or allow others to do so, for the United States Government purposes. The Department of Energy will provide public access to these results of federally sponsored research in accordance with the DOE Public Access Plan.}}

\author{project team}
%\author{K\"ubra Yeter-Aydeniz }
%\email{yeteraydenik@ornl.gov}
%\affiliation{Physics Division, Oak Ridge National Laboratory,
%  Oak Ridge, TN 37831, USA}
%\affiliation{Computational Sciences and Engineering Division, Oak Ridge National %Laboratory,
%  Oak Ridge, TN 37831, USA}




%\author{Alexander F. Kemper}
%\affiliation{Department of Physics, North Carolina State University, Raleigh, North %Carolina 27695, USA}
%\email{akemper@ncsu.edu}

%\author{Raphael C.\ Pooser}
%\email{pooserrc@ornl.gov}
%\affiliation{Computational Sciences and Engineering Division, Oak Ridge National %Laboratory,
%  Oak Ridge, TN 37831, USA}

%\author{Patrick Dreher}
%\affiliation{North Carolina State University, Raleigh, North Carolina 27695, USA}


\date{\today}

\begin{abstract}
 One of the most problematic issues that limits the implementation of applications on today’s Noisy Intermediate Scale Quantum (NISQ) machines are the adverse impacts of qubit errors.  Reliance on standard one and two qubit randomized benchmarking error measurements cannot reliably identify and capture the full impact of these errors.  This is a critical problem because coherent errors may degrade the user application results in an unpredictable manner and may compromise efforts to validate the accuracy of applications implemented on these NISQ quantum processors.  We report here on an in-depth study of this issue using a transverse Ising model Hamiltonian as a sample user application test case.  We demonstrate through a detailed set of measurements how impacts from inter-day and intra-day qubit calibration drift and placement of the quantum circuit on groups of qubits in different physical locations on the processor can lead to coherent and other errors that adversely impact and distort user application results.  This paper also discusses techniques to gain a better understanding of these types of errors and their impact on efforts to validate the accuracy of quantum computations.
\end{abstract}

\maketitle

\import{}{Introduction}
\import{}{Physics Model}
\import{}{Methodology}
\import{}{Data Analysis}
\import{}{Discussion}
\import{}{Summary and Conclusions}

%\bibliographystyle{apsrev4-1}
%\bibliography{refs.bib}

%==========================================================%
%\onecolumn
\bibliography{bibliography}
%\bibliography{main.bbl}
%\twocolumn
%==========================================================%


\import{}{Appendix.tex}

\end{document}

