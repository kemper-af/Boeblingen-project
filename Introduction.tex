

\section{Introduction}
\label{sec:intro}

\begin{itemize}
    \item introduction
\item physics model
\item cycle benchmarking methodology
\item Benchmarking Implementation
\begin{itemize}
    \item Data Acquisition procedure
    \item Measurements
\end{itemize}
\item Data Analysis
\item Discussion
\item summary and conclusions
\item acknowledgments
\item Appendix
\end{itemize}



Key points of the paper
\begin{itemize}
\item hypothesis: both space and time dependent coherent and incoherent errors
\item 	errors not fully characterized by IBM backend properties and random benchmarking – need randomized compiling and cycle benchmarking
\item  general methodology scale independent and device agnostic
\item inter-day re-calibration errors 
\end{itemize}

The advent of Noisy Intermediate Scale Quantum (NISQ)~\cite{Preskill2018quantumcomputingin} era in quantum computing hardware platforms has opened opportunities for many application domains to begin migrating existing algorithms and applications from the digital computing to the quantum computing world.  The earliest quantum computing platforms only consisted of a few qubits and so were very restrictive in the types of quantum algorithms and applications that could be implemented on these platforms .  Because of advancements in quantum computing hardware technology hardware platforms with hundreds to more than a thousand qubits will become available in the next few years.  

%Along with advancements in hardware technologies, software advances have matured to the point where serious prototype applications are now being implemented on these platforms.  

Although the overall hardware capabilities now allow applications to successfully be implemented on these platforms, these machines are still classified as NISQ machines and will likely continue to be so in the intermediate future.  Incoherent errors from qubit interactions with the environment and limitations in the precision of qubit control will continue to produce noise and errors in these systems.  These noise sources will continue to degrade the fidelity of the machine’s output and limit the scalability and reliability of the output.  Measuring and characterizing the various sources of noise in these machines is crucial to understand how these errors will adversely impact a real-world application’s performance on these hardware platforms.

Quantum computing hardware architects have been aware of these issues from the time of the earliest hardware platforms ever produced.  They are continuously focused on developing ever more sophisticated techniques and methods to mitigate the errors on these NISQ platforms {\color{red}{KYA: Cite a few error mitigation papers}}.  

Early attempts to quantify and measure these errors focused on listing single numbers that characterized an coherence of single quibts and the lifetime of single qubit gates.  Additional measurements were recorded that profiled the error rate for two qubits connected together to build two-qubit gates.  Over time morea advanced techniques such as quantum process tomography~\cite{ref:Chuang1997} or gate set tomography~\cite{ref:Merkel2012,ref:Blume-Kohout2013,ref:Blume-Kohout2017} were introduced to fully characterize these quantum processes.  The difficulty in implementing these techniques is that in order to fully characterize a quantum process it requires a number of experiments and digital post-processing resources that grows exponentially with the number of qubits.  Alternatives to such an exhaustive option were ideas of process fidelity measurements.  Such a process fidelity measurement can be determined by randomized benchmarking~\cite{ref:Emerson2005,ref:Dankert2009,ref:Magesan2011}.

\LK{In this paper, we...}

The paper is organized as follows. In Sec.~\ref{sec:model} we describe the physics model used to study the error characterization in NISQ devices. Then in Sec.~\ref{sec:methodology} the data collection process is elaborated which is followed by Sec.~\ref{sec:data-analysis} where we analyzed the data collected and the results are discussed in Sec.~\ref{sec:discussion}. We then finalize the paper with a summary and conclusions in Sec.~\ref{sec:summary_conc}. 